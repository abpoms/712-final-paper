% --------------- 12 POINT FONT -------------------------------
\documentclass[tog]{acmsiggraph}

% --------------- 10 POINT FONT FOR CAPTIONS ------------------

\usepackage{listings}
\usepackage[dvipsnames]{xcolor}
%%% Make the ``BibTeX'' word pretty...

\def\BibTeX{{\rm B\kern-.05em{\sc i\kern-.025em b}\kern-.08em
    T\kern-.1667em\lower.7ex\hbox{E}\kern-.125emX}}

%%% Used by the ``review'' variation; the online ID will be printed on 
%%% every page of the content.

\TOGonlineid{45678}

%%% Used by the ``preprint'' variation.

\TOGvolume{0}
\TOGnumber{0}

% --------------- MATH PACKAGES -------------------------------
\usepackage{amsmath,amsthm,amssymb}

\title{Processing collections of images and videos on big-data frameworks}
\author{Alex Poms, Ravi Teja Mullapudi, \\Suhail Rehman}
\pdfauthor{Alex Poms, Ravi Teja Mullapudi, \\Suhail Rehman}

\begin{document}


\maketitle

\begin{abstract}
\end{abstract}

% --------------- CONTENT -------------------------------------
\section{Introduction}
Images and video already dominate internet traffic and occupy a significant
amount of datacenter storage. The Cisco Visual Networking Index predicts that
IP video traffic will consume over 80\% of total internet bandwidth in 2019. In
2009, Facebook reported 1.5PB of image storage with 25TB added weekly and
550,000 images served per second at peak. This visual data is mostly just
served directly to a human user for viewing. However, recent advances in
computer vision, such as feasible image/object understanding and reconstruction
of 3D structure from image collections, motivate large-scale analysis of images
and videos for deriving insights and information that is not accessible from
other sources. For example, a relief organization could automatically
reconstruct the damage done by a natural disaster by having vehicles with
mounted cameras observe the affected area or a state could perform automatic
demographics estimation by analyzing the public Facebook profile photos of
their residents.

Currently, visual computing applications are few and far between due to the
inherent complexities of working with large visual datasets. The most apparent
of these difficulties is that the massive size of the datasets vastly exceed
the storage and computational capacity of a single machine. Thus any
application must be distributed and parallelized across a large cluster,
managing data transfer and communication between nodes. Unlike traditional
analytics workloads, visual computing applications also require highly tuned
and efficient image processing kernels that take advantage of the heterogeneous
hardware increasingly seen in cloud infrastructure, such as throughput-oriented
accelerators including GPUs and even FPGAs. Having a predefined set of
optimized implementations is not enough because image processing pipelines
require global code restructuring that crosses function boundaries; each
application may require custom image processing code that is composed with
other functionality. Lastly, visual computing applications exhibit tight
coupling between traditional analytics operations, such as those of Spark
\cite{zaharia2012resilient} and MapReduce \cite{dean2008mapreduce}, and
non-streaming operations, like distributed optimization or graph processing.

It is unclear if existing systems can simultaneously satisfy these three
requirements. Systems optimized for analytics operations, like Spark, tend to
sacrifice explicit access to their data collections that prohibit efficient
integration of non-streaming operations. Analytics systems also lack the
ability to specialize user routines to heterogeneous hardware resources. On the
other hand, systems that perform non-streaming operations, like the Parameter
Server \cite{li2014scaling}, generally lack the flexibility in their
programming model to allow direct incorporation of streaming operations like
map, filter, or reduce.

\section{Background}

\subsection{Spark}

\subsection{Legion}

\subsection{SciDB}
SciDB\cite{stonebraker2013scidb} is a computational array DBMS designed to cater to the data storage needs of the scientific community. SciDB is designed to be multi-dimensional, with an array-based data model. It also provides support for data versioning and provenance which is a oft-cited feature requirement of data management systems from scientists.

SciDB arrays are expressed in terms of two basic parameters:  the {\em dimensions} of the array, as well as {\em attributes} of the array. 

An n-dimensional SciDB array has dimensions ($d_{1}$, $d_{2}$,$...$,$d_{n}$). The {\em size} of each dimension is the number of ordered values in that dimension. For example, a 2-dimensional array may have dimensions $i$ and $j$, each with values $(1, 2, 3, ..., 10)$ and $(1, 2, ..., 30)$ respectively. SciDB uses 64-bit integers to represent the values in each dimension, but also support non-integer dimensions such as variable-length strings or floating point integers. Furthermore, SciDB supports the notion of dimensional bounds. When the total value of a dimension is known in advance, the array can be declared with a {\em bounded} dimension. Sometimes, the cardinality of the array may not be known at array creation time. In such cases, the array can be declared with an {\em unbounded} dimension. It must be noted, however, that certain array operations are restricted to arrays with integer bounded dimensions.

In addition to the dimension size attributes, the user must define two parameters for each dimension: the {\em chunk size} and {\em chunk overlap} parameters. These parameters affect the distribution of the array data among the worker nodes and need to be studied with respect to their effect on the performance of the image operations that we intend to deploy in SciDB.

An n dimensional array in SciDB refers to a single cell or element of an array. However, each array in the SciDB array can hold multiple data values known as {\em attributes}. Each data value is referred to as an {\em attribute}, and can be any of the supported data types in SciDB.

Therefore, during the creation of an array in SciDB (analogous to the declaration of a table schema in a relational database), the user must specify:

\begin{itemize}
\item An array name - a simple string which can be used to refer to the array in all operations involving the array.
\item The dimensions of the array. The name and size of each dimension must be declared, with the exception of unbounded dimensions, whose size is represented using the asterisk character (\texttt{*}).
\item At least one attribute for the array. Attributes can be added to an array as the result of an operation in SciDB.
\end{itemize}

An example of an array definition in SciDB is given below:

\begin{lstlisting}[caption=Creating an Array in SciDB, frame=single]
AQL% CREATE ARRAY open <val:double>[I=0:9,10,0,J=0:*,10,0];
\end{lstlisting}

In the example, the name of the array is \texttt{open}, with one attribute named \texttt{val}, of type \texttt{double}. This array consists of two dimensions, \texttt{I} and \texttt{J}. \texttt{I} is a bounded dimension with values ranging from 0 to 9, with chunk size 10 and chunk overlap 0. \texttt{J} another dimension similar to \texttt{I}, but the size of this dimension is unbounded.

SciDB is designed to perform a number of operations on Arrays. An extensive listing of array operations is beyond the scope of this document, the interested reader is referred to the SciDB manual\cite{SciDBManual}. However, the operations that can be performed on an array can be broadly classified into the following: array selection, array operations (such as cross product, joins etc.), aggregation operations (which can return either arrays or scalars) and so on.
\section{Workloads}
Visual data applications contain a diverse set of operations, but a few of them
stand out as particularly important and distinctly core components of the
analysis algorithms they are a part of. In particular, two classes
of operations that seem to span a significant amount of the space of important
operations in visual data applications are: very compute-intensive kernels
applied individually across an entire dataset, and aggregate,
high-communication operations, like computing nearest neighbors or
clustering. For a distributed computing framework to serve as a platform for
creating performant applications that process large collections of images and
videos, it must at least support very efficient and scalable implementations of
the following operations:

\subsection{High-performance map kernel}

\subsection{Clustering}

%\subsection{Filter clustering}

\section{Distributed collections on Legion}

\subsection{}

\subsection{Task scheduling}

\subsection{KMeans Implementation}

\section{Evaluation}

\subsection{Spark vs Legion}

\subsubsection{Partition startup time}

\subsection{SciDB vs MPI}

\section{Discussion}

\subsection{Native code on Spark}

\subsection{Improving distributed collection abstractions}

\subsubsection{Batch size}

\subsubsection{Iterators}

\subsection{Improving SciDB}

\section{Conclusions}

\bibliographystyle{acmsiggraph}
\nocite{*}
\bibliography{final}

\end{document}
%%% Local Variables:
%%% mode: latex
%%% TeX-master: t
%%% End:
